\documentclass[runningheads,a4paper,11pt]{article}

\usepackage{algorithmic}
\usepackage{algorithm} 
\usepackage{array}
\usepackage{amsmath}
\usepackage{amsfonts}
\usepackage{amssymb}
\usepackage{amsthm}
\usepackage{booktabs}
\usepackage{caption}
\usepackage{cite}
\usepackage{comment} 
\usepackage{enumitem}
\usepackage{epsfig} 
\usepackage{fancyhdr}
\usepackage[T1]{fontenc}
\usepackage{geometry} 
\usepackage{graphicx}
\usepackage[colorlinks]{hyperref} 
\usepackage[latin1]{inputenc}
\usepackage{interval}
\usepackage{multicol}
\usepackage{multirow} 
\usepackage{rotating}
\usepackage{setspace}
\usepackage{subfigure}
\usepackage{tabularx}
\usepackage{tikz}
\usepackage{url}
\usepackage{verbatim}
\usepackage{xcolor}
\usepackage{xstring}

\geometry{a4paper,top=2.5cm,left=2cm,right=2cm,bottom=2cm}


\begin{document}
\title{How Consistent are LLMs in Applying Handcrafted Language Features}
\author{Andrei Olar --- andrei.olar@ubbcluj.ro}
\maketitle
\begin{abstract}
    Large language models are sought for solving an increasing number of
    increasingly diverse tasks. Perhaps naturally one such task is text style
    transfer: the task of changing the style of a text while preserving its
    meaning. Besides the natural approach of fine-tuning a large language model,
    a simpler and cheaper alternative should also be investigated: using
    handcrafted language features to direct the large language model.
\end{abstract}

\section{Introduction}\label{introduction}

Large language models have taken one of the main spots in research for quite
some time.
Because language models such as BERT~\cite{devlin2018bert} or GPT~\cite{gpt-2018,gpt2-2019,
gpt3-2020} constantly advance the state of the art for many natural language
processing tasks, it has become natural to want to evaluate these models
performance on ever more tasks.
We know from multiple surveys~\cite{minaee2024llmsurvey,zhao2023survey} and
benchmarks~\cite{papcode2024hellaswag,chiang2024chatbot} that large language
models (especially the more popular ones) are very good at following
instructions.
We can also intuit that training LLMs on diverse data (for instance the
Pile~\cite{gao2020pile}) uniquely qualifies them to produce text in a wide
variety of styles.

Text style transfer is defined as the ``task of transforming the stylistic manner
in which a sentence is written, while preserving the meaning of the original
sentence''~\cite{tst-review-2021}.
This definition can be extended, to entire articles or corpuses containing
multiple articles without fear of text style transfer losing its meaning.
The task of transfering text style is an old preocupation for the natural
language processing and computational linguistics communities which has picked
up interest again in the context of the advancements in the field of deep
learning and particularly since the inception of the transformer architecture.

Handcrafted linguistic features (HLFs) are single numerical values produced by a
uniquely identifiable method on any natural language~\cite{lftk-2023}. More
often than not, these features are easy to compute and intuitively very telling
of the style in which a text is written.

For example, take the number of exclamations in a text.
This particular feature is easier to compute than most other metrics in
computational linguistics and high values of this particular HLF can pretty
reliably signal an inflamatory news writing style, spam or an alert narrative.
The hope that arises from this observation is to describe style sufficiently by
using only HLFs.
<CITATIONS NEEDED>
The appeal of this approach lies within the idempotent nature of most HLFs.
That makes them useful in identifying or verifying styles, too. <CITATIONS NEEDED>

<SOME MORE IDEAS AROUND STYLEPTB, lexical, syntax, semantics,
and thematics, that span the fundamental atomic
transfers that text can undergo (McDonald and
Pustejovsky, 1985; DiMarco and Hirst, 1993).Chomsky 2002,>

From the above ideas, a couple of questions arise naturally.
Firstly, is it possible to influence an LLMs writing style by instructing it to
follow certain HLFs?
If this is possible then it's very financially advantageous to use this approach
instead of fine tuning an LLM.
For one, fine tuning an LLM incurs some loss of generality~\cite{yang2024unveiling}.
And then there's the matter of cost.

Secondly, we ask if it's possible to tell just how well the most prominent LLMs
of today are able to generate text according to instructions concerning HLFs.
If LLMs respond well enough to prompt instructions, this might prove to be an
easy and affordable solution for the text style transfer problem.

\section{Related Work}\label{related}

There is abundent literature on the topic of text style transfer taken from
various perspectives.
For writing this paper, we have mostly consulted literature reviews and
syntheses~\cite{tst-review-2021,tst-survey-2022} on the subject in order to
inform the selection of HLFs that we would be using in our experiments.

The literature on HLFs is extensive and spans most of the field of computational
linguistics.
This work is well referenced and synthesised by Lee and Lee~\cite{lftk-2023}.
To our knowledge there is no work that connects HLFs with LLMs in the manner
described in this paper.
Most other literature focuses on achieving results on connected tasks, such as
assessing text readability~\cite{lee-etal-2021-pushing}.
This paper differs from other endeavours in at least two significant ways.
Firstly, this paper is simply about feeding well crafted user prompts to an LLM
instead of diving into deeper concepts such as the transformer architecture,
deep learning or even neural networks.
Secondly, it focuses squarely on manipulating the style attributes of the text
generated by an LLM using HLFs and nothing else.

That said, text style transfer has two components: transferring style and
retaining meaning.
While this paper focuses on transferring style, there are numerous benchmarks,
studies and reviews that atest to the ability of state of the art LLMs to retain
meaning in rewording exercises. <CITATIONS NEEDED>

- delineate from work in the StylePTB paper (https://aclanthology.org/2021.naacl-main.171.pdf)
- the styleptb paper evaluates style transfer quality which is the subject of
future work

\section{Experiment Design}\label{method}

Our experiment consists of analysing the behavior of an LLM when asked to
generate new text based on a given text and and some instructions derived from
HLF values computed on a benchmark dataset.
To perform the experiment we must know which LLMs we are going to evaluate and
what benchmark datasets we are going to use.
We must also select the HLFs for our experiment.

Before answering those questions, let us outline the experimental process.

The first task in the experiment is to establish a baseline.
We do so by instructing the target LLM to generate text retaining the meaning of
the input text without any further instruction.
We repeat the text generation step 10 times.
For each text generation, we compute the following statistics for each of our
chosen HLFs:

\begin{itemize}
    \item \textbf{min}: the minimum value of the HLF on the generated text;
    \item \textbf{max}: the maximum value of the HLF on the generated text;
    \item \textbf{avg}: the mean value of the HLF on the generated text;
    \item \textbf{var}: the variance of the HLF values on the generated text.
\end{itemize}

A second level baseline is then obtained by following the above process with a
prompt that builds upon the first one by adding the instruction to use example
text from our benchmark dataset of choice.
The examples are five texts chosen randomly once for each benchmark dataset.
These example texts are the same for all evaluated LLMs.

The next step is to devise HLF prompts derived from both of the baseline
prompts.
The HLF prompts contain additional instructions to generate text that would
comply with certain HLF values.
These HLF values are obtained by computing the statistics outlined above on
the entirety of the target benchmark dataset.

We then repeat the text generation and statistics computation with each of the
HLF prompts in the same way as we did for their corresponding baselines.

To answer our first question, we compare the first baseline with the results
obtained by using the corresponding modified HLF prompt.
Our reasoning is that if the LLM were influenced by the HLF prompt, the output
would show different HLF values compared to the baseline.
This assumption is rooted in the work on detecting LLM generated text.<CITATION NEEDED>

To answer the question of how much LLMs are influenced by being prompted with
instructions about HLFs, we turn to the second level baseline.
Given that the LLM now also has example articles, it can freely immitate those
articles when generating the text.
If there is a consistent difference between the HLF and non-HLF prompt, this
difference can only stem from the HLF instructions.
Therefore, how much the HLF values differ between the second level baseline and
the corresponding HLF generated texts is strictly due to the HLF instructions.

One final observation before diving into details is that we can use the variance
computed for HLFs on the benchmark datasets to our advantage.
This statistic shows us whether the LLM is at least roughly compliant with the
instructions provided in the HLF prompts.
To understand how let us denote an HLF computed on a text generated by using an
HLF prompt with $HLF_g$.
Similarly, we denote the average and the variance of the same HLF on the
benchmark dataset with $avg(HLF_b)$ and $var(HLF_b)$, respectively.
If we consistently notice that
    \[HLF_g - var(HLF_b) \leq avg(HLF_b) \leq HLF_g + var(HLF_b)\]
then we consider that the LLM consistently follows the HLF instructions in the
prompt.

\subsection{LLM Selection}\label{llm-selection}

In order to perform our experiment we must first choose proper candidates.
Since the interaction with the model happens only via prompts, we may use models
with any type of licensing.
Our choice will be the top model from each of the highest scoring five different
major vendors listed in the Chatbot Arena benchmark~\cite{chiang2024chatbot}.

<TODO - make a table with model specs: no params, context size, ELO, license>
OpenAI - gpt-4o
Google - Gemini-1.5-Pro-API-0514
Anthropic - Claude 3 Opus
Meta - Llama-3-70b-Instruct
Cohere - Command R+

\subsection{HLF Selection}\label{hlf-selection}

On the other hand, we need to look at which handcrafted linguistic features are
relevant for our experiment.
In our opinion, good handcrafted linguistic features for the task of text style
transfer are those features that can reasonably be thought to affect one of the
coordinates by which people judge the style of text.
These coordinates are well enumerated both in~\cite{tst-review-2021} and
in~\cite{tst-survey-2022}.

In order to build upon the previous work done regarding computing handcrafted
linguistic features, we will rely only on the features available in
LFTK~\cite{lftk-2023}.

Based on the above two selection criteria, we have selected the handcrafted
linguistic features in <INSERT TABLE HERE>.

\subsection{DataSet Selection}\label{ds-selection}

- Are there benchmark datasets for text style transfer? (STYLEPTB + Yelp Reviews)

\subsection{Input Prompt and Text}\label{input-text}

- link to text from within the corpus
- link to text from outside the corpus
- link to each system prompt (2 baselines, 2 HLF prompt templates)

\section{Experiment Results}

\subsection{Graph Interpretation}

Ox - the number of the generation 1..10
Oy - HLF value
Continuous grey line - baseline metric
continuous grenat line - hlf metric
grey around grenat line - corpus target zone
dotted black lines - corpus min, max
continuous black line - corpus avg

\subsection{Baseline}

Yelp Results

STYLEPTB Results

\subsection{Context Aware}

Yelp Results
StylePTB Results

\section{Discussion}

Discuss experiment results here.

\section{Conclusions and Future Work}

Yes, you can instruct at least some of the LLMs with HLFs.

compute BLEU, meteor, cider on styleptb using these LLMs and the prompt to
evaluate whether we actually achieve style transfer because of/in spite of HLF
instructions.
Human survey (amazon mechanical turk) on whether the text generations are good
or not. Likert scale (0 no good, 10 matches style perfectly)

\bibliographystyle{plain}
\bibliography{references}
\end{document}